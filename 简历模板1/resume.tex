\documentclass{resume}
\usepackage{zh_CN-Adobefonts_external} 
\usepackage{linespacing_fix}
\usepackage{cite}
\usepackage{hyperref}
\hypersetup{
    colorlinks=true,
    linkcolor=cyan,
    filecolor=magenta,  
    urlcolor=blue,
}

\begin{document}
\pagenumbering{gobble}

%***"%"后面的所有内容是注释而非代码,不会输出到最后的PDF中
%***使用本模板,只需要参照输出的PDF,在本文档的相应位置做简单替换即可
%***修改之后,输出更新后的PDF,只需要点击Overleaf中的“Recompile”按钮即可

%在大括号内填写其他信息,最多填写4个,但是如果选择不填信息,
%那么大括号必须空着不写,而不能删除大括号。
%\otherInfo后面的四个大括号里的所有信息都会在一行输出
%如果想要写两行,那就用两次这个指令(\otherInfo{}{}{}{})即可


%***********个人信息**************
\MyName{蒋昱铭}
\sepspace
\SimpleEntry{XXX国家重点实验室}
\SimpleEntry{mingjiangyu1@qq.com}
\SimpleEntry{19802055081}
\SimpleEntry{个人主页:\href{https://doublelll3.ml}{https://doublelll3.ml}}

%************照片**************
%照片需要放到images文件夹下,名字必须是you.jpg,注意.jpg后缀(可以去resume.cls第101行处修改),如果不需要照片可以不添加此行命令
%0.15的意思是,照片的宽度是页面宽度的0.15倍,调整大小,避免遮挡文字
\yourphoto{0.14}

%***********教育背景**************
\section{教育背景}
%***第一个大括号里的内容向左对齐,第二个大括号里的内容向右对齐
%***\textbf{}括号里的字是粗体,\textit{}括号里的字是斜体
\datedsubsection{\textbf{华南理工大学(985)},软件工程,\textit{本科}}{2023.9 - 至今}
% \datedsubsection{\textbf{xxxx大学},xxxxxx及控制,\textit{本科}}{2014.9 - 2018.7}
\begin{itemize}
  \item 专业基础课程学习扎实,数据结构与算法(91),算法设计与分析(89),离散数学(90),计算机组成原理(94)。 
  \item CSAPP, CS149, CMU15-445
\end{itemize}

%***********过往经历**************
\section{项目经历}
\datedsubsection{\textbf{C++17开发的高性能磁盘缓冲区}}{2025.9 - 2025.10}
\begin{itemize}
  \item 实现了自适应缓存替换算法(Arc缓存替换算法) ,避免一过性访问造成的高频访问数据丢失。同时支持动态调整缓冲区参数设置,以适应当前工作负载的 “时间局部性” 和 “空间局部性” 特征,提高缓存命中率。
  \item 实现异步磁盘 I/O,将磁盘刷写吞吐从 A MB/s 提升至 B MB/s(+Z\%)(\textbf{TODO:}补充 A/B/Z 的基准数据与测试环境描述)。
  \item 运用 RAII 范式,实现线程安全的页读写保护器。
  \item 通过 {perf}/Flamegraph 定位热点,优化关键路径;p99 延迟下降 N\%(\textbf{TODO:}补充 N 与样本规模)。
\end{itemize}

\datedsubsection{\textbf{wget 命令行工具 (Rust)} \href{https://github.com/mistcoversmyeyes/wget-DragonOS}{\scriptsize[GitHub]}}{2025.7 - 2025.8}
\begin{itemize}
  \item 从零开始实现了一个功能对标 `wget` 的命令行下载工具,通过直接操作 TCP 套接字(`std::net::TcpStream`)发送 HTTP GET/HEAD 请求,手动解析 HTTP 响应报文,实现了对网络协议底层的控制。
  \item 设计并实现了一套模块化的架构,将项目解耦为命令行接口(`cli`)、网络核心(`web`)、文件写入(`FileWriter`)和日志(`log`)四个核心模块,提升了代码的可维护性和扩展性。
  \item 使用 {clap} 库构建了健壮的命令行接口,支持 URL、指定输出文件名({-O})和调试模式({-d})等核心参数。
  \item 通过发送 HTTP HEAD 请求预先获取 `Content-Length`,并利用文件句柄的 `seek` 操作,为未来实现断点续传功能奠定了基础。
  \item \textbf{TODO:}支持 HTTPS/TLS({rustls})、3xx 重定向与超时/重s试(指数退避),提升长文件下载稳定性(失败率由 X\% \textrightarrow{} Y\%)。
  \item \textbf{TODO:}实现断点续传(Range/206)、并发分片与限速;预计吞吐 A \textrightarrow{} B MB/s,p95 时延降低 N\%。
  \item \textbf{TODO:}引入结构化日志与指标({tracing}/Prometheus),完善基准测试矩阵与 GitHub Actions 多平台构建发布。
\end{itemize}

% \datedsubsection{\textbf{xxx研究院(实习)},xxxx实习生}{2020.1 - 2020.5}
% \Content
% {负责xxx项目,对xxx进行智能诊断。}
% {搭建xxx网站,结合xxx,利用xxx,基于xxx实现xxx。}
% {1.完成xxx;2.基于xxx,算法准确率在85\%左右。}

% \datedsubsection{\textbf{xxx大学研究生课题},xxx}{2018.3 - 至今}
% \Content
% {针对xxx的不足,研究一种基于xxx对xxx进行xxx的方法。}
% {基于xxx建立xxx模型,使xxx,基于xxx,设计一种xxx。}
% {1.建立了xxx,并在xxx得到了验证;2.对xxx,算法准确率在95\%以上;3.对xxx可以自主学习,从而xxx。}

% \datedsubsection{\textbf{大学生创新创业项目},xxx}{2016.3 - 2017.3}
% \Content
% {针对xxx,基于xxx,加入xx实现了xxx的功能(主要负责人)。}
% {搭建xxx,利用xxx的输出,实现对xxx的控制。}
% {1.实现了xxx;2.获得了xxx奖等荣誉。}

\section{专业技能}
\datedsubsection{\textbf{编程语言}:\textbf{C++17/14/11}(STL/并发/内存模型/RAII)、\textbf{Rust}(所有权/借用/生命周期、错误处理、async/await)}{}
\datedsubsection{fjdsia f}
\sepspace

\section{奖励荣誉}
% \datedsubsection{\textbf{比赛方面}:}{}
% \datedsubsection{xxx}{2020}
% \datedsubsection{xxx}{2020}
% \datedsubsection{xxx}{2017}
% \datedsubsection{\textbf{论文方面}:xxx}{2020}
% \datedsubsection{\textbf{运动方面}:xxx}{2018、2019}

\end{document}